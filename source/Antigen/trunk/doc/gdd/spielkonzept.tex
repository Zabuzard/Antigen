\subsection{Zusammenfassung}
%Hier ist ein kurzer einleitender Text evtl. in Verbindung mit einem Bild gefragt. Ziel ist es, das zu erstellende Spiel in kurzen Sätzen zu erklären und die Grundidee zu erläutern. Für die Zusammenfassung kann man sich an den "Klappentexten" auf der Rückseite von Spieleverpackungen orientieren. Der Text darf als einziger im GDD auch reißerisch und dramatisch sein (abgesehen vom Screenplay).

Eine Geschichte, die uns alle betrifft: die Realität. Tag für Tag wehrt sich der
menschliche Körper gegen alle Arten von Angriffen. Dabei sind häufig diejenigen
Bedrohungen am gefährlichsten, die man nicht sehen kann. Trotzdem gibt
es immer wieder Menschen, die diesen gefahrlos begegnen können, während andere
an ihnen zugrunde gehen. Doch wie geht das alles vor sich? \enquote{\name} ist ein
2D-Echtzeitstrategiespiel, das bewusst an die Realität angelehnt ist. Der
Spieler hat die Möglichkeit, sich verschiedener Abwehrmechanismen zur Bekämpfung
der drohenden Krankheiten zu bedienen. Werden diese geschickt kombiniert, kann
es ihm gelingen, die angreifenden Viren und Bakterien in die Flucht zu
schlagen.

\subsection{Alleinstellungsmerkmal}
%Was hebt dieses Spiel von der Masse ab? Wodurch versucht man, den Spieler (und den Kunden) zu begeistern? Das Alleinstellungsmerkmal ist das Merkmal des Spiels, welches es einzigartig macht. Ein Alleinstellungsmerkmal kann sowohl ein Feature, als auch ein gesamtes Konzept des Spiels sein.

Durch sein flexibles Mutationssystem erlaubt Antigen es dem Spieler, Zellen
nach Belieben anzupassen. Mutationen können alle Eigenschaften seiner Zellen,
von Lebenspunkten über Virenresistenz bis zur Sichtweite, verändern, sodass am
Ende eine für die Bekämpfung des Gegners maßgeschneiderte Einheit steht. Doch
das gleiche System steht auch der KI zur Verfügung, die versuchen wird, die
Anpassungen des Spielers zu unterlaufen. Außerdem gehen die meisten
Verbesserungen einer Eigenschaft mit Verschlechterungen in anderen Bereichen
einher, sodass die spezialisiertesten Zellen auch am anfälligsten für Konter
sind.
