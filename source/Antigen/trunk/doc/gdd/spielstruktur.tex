\subsubsection{Startkonfiguration}

Zu Beginn eines jeden Spiels kontrolliert der Spieler eine geringe Anzahl
Stammzellen, die KI eine geringe Anzahl Bakterien und Viren. Diese befinden
sich räumlich voneinander getrennt im Blutkreislauf.

\subsubsection{Spielablauf}

Die KI beginnt direkt nach Spielbeginn, mithilfe ihrer Viren Rote
Blutkörperchen zu übernehmen und ihre Bakterien mithilfe von Zellteilung zu
vermehren. Gleichzeitig beginnt der Spieler, Stammzellen und Riesenfresszellen
zu produzieren.

Der Spieler sollte sich möglichst früh einen Überblick über die Zusammensetzung
der gegnerischen Einheiten (Bakterien oder Viren, Anzahl verschiedener Stämme)
verschaffen. Mithilfe dieser Informationen kann er entscheiden, wie er seine
Abwehr zusammensetzen möchte.

Während er Rote Blutkörperchen produziert und seine Zellen vor den Angriffen
der gegnerischen Einheiten schützt, versucht der Spieler, Antigene aus
Viren und Bakterien zu extrahieren. Ist ihm dies gelungen, kann er mit der
Produktion spezifischer Abwehr beginnen, die erstmals eine nachhaltige
Verteidigung ermöglicht.

Mithilfe der spezifischen Abwehr kann der Spieler nun beginnen, die
gegnerischen Einheiten zurückzudrängen. Gleichzeitig muss er konstant darauf
achten, genügend unspezifische Abwehr für neu mutierte Viren- und
Bakterienstämme zur Verfügung zu haben und gegen diese Stämme möglichst zügig
spezifische Abwehr zu produzieren.

Gelingt dies dem Spieler, so kann er allmählich die Oberhand gewinnen und
schließlich alle Viren und Bakterien besiegen.

\subsubsection{Gewinn- und Verlustbedingungen}

Der Spieler gewinnt, sobald alle Viren und Bakterien aus dem Blutkreislauf
entfernt sind. Er verliert, wenn entweder alle eigenen Zellen zerstört werden
oder die Anzahl Roter Blutkörperchen im Blutkreislauf unter einen bestimmten
Schwellwert sinkt.

\subsubsection{Strategie}

\minisec{Ressourcen}

Zellen erfüllen in \name{} gleichzeitig die Funktionen von Einheiten und
Produktionsmitteln in vergleichbaren Echtzeitstrategiespielen, da sie sich
in den meisten Fällen teilen und somit neue Zellen produzieren können. Dabei
stehen Zellen, die sich gerade teilen, nicht für andere Aufgaben zur Verfügung;
die zwei Funktionen einer Zelle schließen sich also gegenseitig aus.

\minisec{Spezifische vs. unspezifische Abwehr}

Zu Beginn des Spiels stehen dem Spieler nur die Einheiten der unspezifischen
Abwehr, Riesenfresszellen, zur Verfügung. Um die ungleich stärkeren Einheiten
der spezifischen Abwehr -- T-Zellen und Antikörper -- herzustellen, müssen
zunächst aufwändig die Antigene der Viren und Bakterien extrahiert werden, was
über einen relativ langen Zeitraum viele Ressourcen bindet. Außerdem kommen
durch Mutation neue Antigene hinzu, gegen die die bis dato produzierte
spezifische Abwehr wirkungslos ist. Der Spieler muss somit permanent
zwischen der unmittelbar wirksamen aber langfristig nicht aufrechtzuerhaltenden
unspezifischen und der zeitintensiven, mehr Produktionsmittel bindenden, aber
langfristig mächtigeren spezifischen Abwehr abwägen und seine Produktionsmittel
korrekt zuweisen.

Hiermit verbindet sich auch die Notwendigkeit, über die Konfiguration der
gegnerischen Viren und Bakterien möglichst genau im Bilde zu sein, um die
optimale Verteidigungsstrategie entwickeln zu können. Die entsprechende
Aufklärungsarbeit bindet weitere Zellen, die nicht für Angriff oder
Verteidigung zur Verfügung stehen oder Gefahr laufen, in gegnerischem
Territorium getötet zu werden.

\minisec{Abwehr vs. Rote Blutkörperchen}

Da Stammzellen sowohl Abwehrzellen als auch Rote Blutkörperchen, deren Mangel
zur Niederlage führt, produzieren, muss der Spieler stets so wenige
Rote Blutkörperchen wie möglich, aber so viele wie nötig herstellen. Da
er über die Bewegungen des Gegners normalerweise nicht vollständig informiert
ist und die Produktion Roter Blutkörperchen Zeit benötigt, muss er einen
gewissen Puffer an Roten Blutkörperchen vorhalten. Die Wahl der Größe dieses
Puffers gelingt besser, je mehr Informationen über den Zustand der Spielwelt
vorliegen.

\minisec{Verteidigung eigener Einheiten vs. Vernichtung des Gegners}

Die Produktionseinheiten des Spielers (Stamm- und B-Zellen) sind relativ
langsam, können sich nur schlecht verteidigen und benötigen zur Reproduktion
viel Zeit. Sie müssen deswegen vor Angriffen der Viren und Bakterien geschützt
werden. Gleichzeitig ist es von Vorteil, die gegnerischen Einheiten so früh wie
möglich zu stören und ihre Reproduktion durch Zellteilung bzw. Übernahme
freundlicher Zellen zu verhindern. Hieraus ergibt sich die Notwendigkeit, den
Gegner zu attackieren und dadurch die eigenen Produktionseinheiten für Angriffe
verwundbar zu machen. Die gezielte Mutation von Kampfeinheiten zur Verbesserung
der Eigenschaft Geschwindigkeit (im Allgemeinen auf Kosten anderer
Eigenschaften) und die Einholung von Informationen über die Bewegungen des
Gegners können diesen Nachteil mitigieren.

\minisec{Ressourcenproduktion vs. Kampfeinheiten}

Stammzellen und Kampfeinheiten werden beide durch Stammzellen produziert.
Insbesondere zu Beginn des Spiels muss der Spieler somit abwägen, ob er
lieber frühzeitig den Gegner attackieren oder seine Ressourcenproduktion
steigern möchte.

\minisec{Spezialisierung vs. Generalisten}

Mutationsfelder erlauben es sowohl dem Spieler als auch der KI, eigene
Einheiten zu spezialisieren. Dieser Prozess ist allerdings zeitaufwändig und
birgt die Gefahr, dass sich durch die Mutation die Eigenschaften der mutierten
Einheiten verschlechtern statt verbessern, weswegen es manchmal nötig sein
kann, zugunsten kurzfristiger Vorteile auf die Spezialisierung zu verzichten.

Weiterhin können überspezialisierte Einheiten große Schwächen gegen bestimmte
Einheiten aufweisen, indem beispielsweise eine Riesenfresszelle mit hoher
Angriffsstärke, aber niedriger Virenresistenz Angriffen durch Viren
schutzlos ausgeliefert ist. Deswegen kann es von Vorteil sein, Generalisten zu
bevorzugen, die gegen jede gegnerische Einheit mäßig effektiv sind.
