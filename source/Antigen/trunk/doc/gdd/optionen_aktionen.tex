\subsubsection{Zellmodi}
\label{sec:modi}

Zellen befinden sich zu jedem Zeitpunkt in einem von vier Modi und verhalten
sich ohne weitere Interaktion des Nutzers ihrem aktuellen Modus entsprechend.

\begin{description}
  \item[Angriff] Die Zelle greift die erste gegnerische Zelle, die ihren
    Sichtradius betritt, an und verfolgt sie, bis die gegnerische Zelle tot
    oder außer Sichtweite ist. Kommen mehrere Zellen für die Auswahl des
    Ziels infrage, so wird die räumlich nächste gewählt.
  \item[Stellung halten] Die Zelle bewegt sich nicht und reagiert nicht
    auf gegnerische Zellen in ihrer Sichtweite, außer wenn diese in
    Angriffsreichweite sind. In diesem Fall attackiert die Zelle die
    gegnerische Zelle, bis jene tot oder nicht mehr in Angriffsreichweite
    ist.
  \item[Treiben lassen] Die Zelle bewegt sich mit dem Blutfluss und
    verhält sich dabei passiv, reagiert also nicht auf Zellen in ihrer
    Umgebung.
  \item[Zellteilung] Die Zelle teilt sich und produziert kontinuierlich,
    in einem von ihrer Zellteilungsgeschwindigkeit bestimmten Intervall,
    neue Zellen. Dabei verhält sie sich passiv, reagiert also nicht auf
    Zellen in ihrer Umgebung, und bewegt sich nicht.
\end{description}

\subsubsection{Aktionen}
\label{sec:aktionen}

Im Folgenden werden alle Aktionen, mit denen der Spieler oder die KI den
Zustand des Spiels beeinflussen können, dargestellt.

\newcounter{usecasect}

\newcommand*{\oder}{\textbf{oder} }
\newcommand*{\actionentry}[5]{%
  \stepcounter{usecasect}
  \minisec{\theusecasect . #1}%
%
  \begin{tabular}{|l|p{9cm}|}%
    \hline%
    Akteur & #2\\%
    Startbedingung & #3\\%
    Schlussbedingung & #4\\%
    \hline%
  \end{tabular}%
%
  #5%
}

\actionentry
{Eine Zelle auswählen}
{Spieler}
{--}
{Die Einheit ist ausgewählt. Alle anderen Einheiten sind abgewählt.}
{
  \begin{compactenum}
    \item Der Spieler linksklickt auf eine auswählbare Einheit.
    \item Die Einheit wird ausgewählt.
  \end{compactenum}
}

\actionentry
{Mehrere Zellen auswählen}
{Spieler}
{--}
{
  Alle auswählbaren Einheiten im Rechteck sind ausgewählt. Alle anderen
  Einheiten sind abgewählt.
}
{
  \begin{compactenum}
    \item Der Spieler zieht ein Auswahlrechteck auf, in dem sich mindestens
      eine auswählbare Einheit befindet.
    \item Die auswählbaren Einheiten im Auswahlrechteck werden ausgewählt.
  \end{compactenum}
}

\actionentry
{Einheiten zu erreichbarem Punkt bewegen}
{Spieler oder KI}
{Es sind kontrollierbare Einheiten ausgewählt.}
{Die ausgewählten Einheiten befinden sich am Zielpunkt.}
{
  \begin{compactenum}
    \item Der Akteur klickt mit der rechten Maustaste auf einen begehbaren
      Punkt in der Welt.
    \item Ausgewählte Einheiten, die sich vorher im Treiben- oder
      Zellteilungsmodus befanden, wechseln in den Angriffsmodus.
    \item Die ausgewählten Einheiten bewegen sich ausgehend von ihrer
      Position auf dem kürzesten Weg auf den Punkt zu.
    \item Die ausgewählten Einheiten erreichen den Zielpunkt.
  \end{compactenum}
}

\actionentry
{Versuch, Einheiten zu nicht erreichbarem Punkt zu bewegen}
{Spieler oder KI}
{Es sind kontrollierbare Einheiten ausgewählt.}
{
  Die ausgewählten Einheiten befinden sich am dem Zielpunkt am nächsten
  liegenden erreichbaren Punkt.
}
{
  \begin{compactenum}
    \item Der Akteur klickt mit der rechten Maustaste auf einen nicht begehbaren
      Punkt in der Welt.
    \item Ausgewählte Einheiten, die sich vorher im Treiben- oder
      Zellteilungsmodus befanden, wechseln in den Angriffsmodus.
    \item Die ausgewählten Einheiten bewegen sich ausgehend von ihrer
      Position auf den dem gewählten Punkt am nächsten liegenden erreichbaren
      Punkt zu.
    \item Die ausgewählten Einheiten erreichen den dem gewählten Punkt am
      nächsten liegenden erreichbaren Punkt.
  \end{compactenum}
}

\actionentry
{Modus einer Zelle umschalten}
{Spieler}
{
  Es sind kontrollierbare Zellen ausgewählt, die den gewählten Modus
  unterstützen.
}
{Die Zellen verhalten sich fortan dem gewählten Modus entsprechend.}
{
  \begin{compactenum}
    \item Der Spieler betätigt den Knopf für einen der Zellmodi.
    \item Die Zellen wechseln in den entsprechenden Modus.
  \end{compactenum}
}

\actionentry
{Gegner angreifen}
{Spieler oder KI}
{Es sind Zellen ausgewählt, die die gegnerische Einheit angreifen können.}
{
  Die gegnerische Einheit ist tot \oder die gegnerische Einheit befindet
  sich in einem Bereich der Spielwelt außerhalb der Sichtradien aller
  vom Spieler kontrollierten Einheiten.
}
{
  \begin{compactenum}
    \item Der Akteur rechtsklickt auf eine gegnerische Einheit.
    \item Ausgewählte Einheiten, die sich vorher im Treiben- oder
      Zellteilungsmodus befanden, wechseln in den Angriffsmodus.
    \item Die ausgewählten Einheiten bewegen sich zur gegnerischen Einheit.
    \item Die ausgewählten Einheiten ziehen der gegnerischen Einheit
      Lebenspunkte entsprechend ihrer Angriffsstärke ab.
    \item Bewegt sich die gegnerische Einheit, so folgen ihr die ausgewählten
      Einheiten.
    \item Die Lebenspunkte der gegnerische Einheit erreichen den Nullpunkt
      und sie stirbt \oder die gegnerische Einheit bewegt sich aus der
      Sichtweite des Spielers hinaus.
  \end{compactenum}
}

\actionentry
{Infektion}
{KI}
{
  Ein Virus ist ausgewählt. Die Infektionsstärke des Virus übersteigt die
  Virenresistenz der gegnerischen Zelle.
}
{
  Die gegnerische Zelle ist nun eine infizierte Zelle. Das Virus ist
  tot.
}
{
  \begin{compactenum}
    \item Die KI weist dem ausgewählten Virus eine anzugreifende gegnerische
      Zelle zu.
    \item Das Virus bewegt sich zur Zelle.
    \item Das Virus infiziert die Zelle: die Zelle wird zu einer infizierten
      Zelle. Das Virus stirbt.
  \end{compactenum}
}

\actionentry
{Zellproduktion}
{Spieler}
{Es ist mindestens eine Zelle, die sich teilen kann, ausgewählt.}
{
  Die ausgewählten Zellen produzieren kontinuierlich Zellen der ausgewählten
  Art.
}
{
  \begin{compactenum}
    \item Der Spieler wählt im Produktionsmenü aus, welche Zellart bei
      der Zellteilung produziert werden soll.
    \item Der Spieler aktiviert den Zellteilungsmodus der ausgewählten
      Zellen.
    \item Die ausgewählten Zellen beginnen, Zellen der gewählten Art zu
      produzieren.
  \end{compactenum}
}

\actionentry
{Antigen extrahieren}
{Spieler}
{Es ist eine Riesenfresszelle, die kein Antigen besitzt, ausgewählt.}
{
  Die gegnerische Zelle ist tot und die Riesenfresszelle besitzt das Antigen
  der gegnerischen Zelle.
}
{
  \begin{compactenum}
    \item Der Spieler lässt die Riesenfresszelle eine gegnerische Zelle
      (Bakterium oder Virus) angreifen.
    \item Die Riesenfresszelle tötet die gegnerische Zelle.
    \item Die Riesenfresszelle übernimmt automatisch das Antigen der
      gegnerischen Zelle.
  \end{compactenum}
}

\actionentry
{Antigen übernehmen}
{Spieler}
{Es ist eine antigenverarbeitende Zelle ausgewählt.}
{
  Die ausgewählte Zelle hat das Antigen übernommen und die Riesenfresszelle
  besitzt kein Antigen mehr.
}
{
  \begin{compactenum}
    \item Der Spieler rechtsklickt auf eine Riesenfresszelle, die ein Antigen
      besitzt.
    \item Die ausgewählte Zelle bewegt sich zur Riesenfresszelle.
    \item Die ausgewählte Zelle erreicht die Riesenfresszelle.
    \item Die ausgewählte Zelle übernimmt das Antigen von der Riesenfresszelle.
  \end{compactenum}
}

\actionentry
{Antigen übergeben}
{Spieler}
{Es ist eine Riesenfresszelle mit extrahiertem Antigen ausgewählt.}
{
  Die antigenverarbeitende Zelle hat das Antigen übernommen und die
  Riesenfresszelle besitzt kein Antigen mehr.
}
{
  \begin{compactenum}
    \item Der Spieler rechtsklickt auf eine antigenverarbeitende Zelle.
    \item Die ausgewählte Riesenfresszelle bewegt sich zur
      antigenverarbeitenden Zelle.
    \item Die ausgewählte Riesenfresszelle erreicht die antigenverarbeitende
      Zelle.
    \item Die antigenverarbeitende Zelle übernimmt das Antigen der
      Riesenfresszelle.
  \end{compactenum}
}

\actionentry
{Kamerakontrolle}
{Spieler}
{Die Kamera kann sich weiter in Richtung des Cursors verschieben.}
{Die Kamera hat sich in Richtung des Cursors verschoben.}
{
  \begin{compactenum}
    \item Der Spieler bewegt den Cursor an einen der Ränder des Spielfensters.
  \end{compactenum}
}

\actionentry
{Zoom}
{Spieler}
{Die Kamera ist nicht vollständig herein- bzw. herausgezoomt.}
{Die Kamera ist weiter herein- bzw. herausgezoomt.}
{
  \begin{compactenum}
    \item Der Spieler bewegt das Mausrad nach vorn bzw. hinten.
  \end{compactenum}
}

\actionentry
{Pausieren}
{Spieler}
{Das Spiel ist nicht pausiert.}
{Das Spiel ist pausiert.}
{
  \begin{compactenum}
    \item Der Spieler betätigt den Pauseknopf.
    \item Das Pausemenü öffnet sich.
  \end{compactenum}
}

\actionentry
{Pause beenden}
{Spieler}
{Das Spiel ist pausiert.}
{Das Spiel läuft weiter.}
{
  \begin{compactenum}
    \item Der Spieler betätigt im Pausemenü den \enquote{Weiter}-Knopf.
    \item Das Pausemenü schließt sich.
  \end{compactenum}
}

\subsubsection{Abbruch von Aktionen}

Alle oben aufgeführten Aktionen können wie folgt abgebrochen werden:

\begin{enumerate}
  \item Abbruch durch den Spieler oder die KI: Gibt der Akteur einer
    Zelle, während diese eine Aktion ausführt, einen neuen Befehl, so
    bricht die Zelle die Aktion ab und beginnt mit der Ausführung des
    Befehls.
  \item Abbruch durch Tod: Stirbt die Zelle, die eine Aktion ausführt,
    oder die Zelle, die Ziel einer Aktion (beispielsweise eines Angriffs) ist,
    so wird die Aktion abgebrochen.
\end{enumerate}

\subsubsection{Effekte der Zellteilung}
\label{sec:zellteilung}

Bei der Zellteilung werden je nach produzierender und produzierter Zellart die
Eigenschaften der produzierten Zelle unterschiedlich gesetzt. Dieser Abschnitt
gibt einen Überblick über die involvierten Spielmechaniken.

\minisec{Reproduktion}

Bei der Reproduktion einer Zelle (d.h. produzierende und produzierte Zelle
sind von der gleichen Art) werden die Eigenschaften der produzierenden an
die produzierte Zelle vererbt. Dabei besteht eine geringe Wahrscheinlichkeit,
dass die produzierte Zelle mutiert und dadurch ihre Eigenschaftswerte gegenüber
denen der produzierenden Zelle verändert.

\minisec{Produktion durch ungleichartige Zelle}

Bei der Produktion einer Zelle durch eine Zelle anderer Art (z.B. einer
Riesenfresszelle durch eine Stammzelle) wird die Mutation einer Eigenschaft
der produzierenden Zelle anteilsmäßig weitergegeben.

Beispiel: Hat eine Stammzelle durch Mutation 5\% Lebenspunkte gegenüber ihrem
Startwert 70 hinzugewonnen, so erhalten auch alle von ihr produzierten Zellen
5\% mehr Lebenspunkte gegenüber ihren Startwerten (siehe
\secref{sec:zellen}).

Zellen, die andersartige Zellen produzieren können, besitzen Werte für
alle Eigenschaften der produzierten Zellen, auch wenn sie diese selbst
nicht verwenden (z.B. Angriffsstärke bei Stammzellen). Diese Werte werden
wie andere Eigenschaftswerte auch durch Zellteilung und Mutation modifiziert.

\minisec{B-Zellen und Antikörper}

Sowohl B-Zellen als auch Antikörper besitzen eine sogenannte Debuff-Tabelle,
die jeder Eigenschaft einen Wert zuordnet. Attackiert ein Antikörper eine
gegnerische Zelle, so wird jede Eigenschaft der attackierten Zelle um den
in der Debuff-Tabelle angegebenen Wert verringert.

Bei der Produktion eines Antikörpers durch eine B-Zelle wird die Debuff-Tabelle
der B-Zelle an den produzierten Antikörper weitergegeben. Mit geringer
Wahrscheinlichkeit kann sie sich durch Mutation positiv oder negativ ändern.

Bei der Produktion einer B-Zelle durch eine andere B-Zelle wird die
Debuff-Tabelle der produzierenden auf die produzierte B-Zelle übertragen
und kann sich durch Mutation ändern.

Bei der Produktion einer B-Zelle durch eine Stammzelle wird die Debuff-Tabelle
der B-Zelle mit dem Standardwert von 20 für jede Eigenschaft initialisiert.

\subsubsection{Sichtradius und Fog of War}
\label{sec:fow}

Als Nebeneffekt jeder Aktion, die die Position einer Einheit verändert,
verändert sich der Bereich der Spielwelt, auf dem der Spieler gegnerische
Einheiten und rote Blutkörperchen sehen kann. Dieser Bereich ist die
Vereinigung der Sichtradien aller vom Spieler kontrollierten Einheiten.
Außerhalb der Sichtradien kann der Spieler zwar den Verlauf der Blutbahnen,
nicht aber gegnerische Einheiten und rote Blutkörperchen sehen.
